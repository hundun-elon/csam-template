\documentclass[a4paper,twoside,12pt]{report}
% Richard Klein (2020,2021)

% Include Packages
%\usepackage[a4paper,inner=3.5cm,outer=2.5cm,top=2.5cm,bottom=2.5cm]{geometry}  % Set page margins
\usepackage{fullpage}
\usepackage{float}                  % Allows 'Here and Only Here' [H] for Floats
\usepackage{url}                    % \url{} command
\usepackage{charter}                  % Set font to Times
\usepackage{graphicx}               % \includegraphics
\usepackage{subfigure}              % Allow subfigures
\usepackage{amsmath}
\usepackage{amssymb}
\usepackage{amsthm}
\usepackage{booktabs}
\usepackage{parskip}
\usepackage[all]{nowidow}
\setnoclub[2]
\setnowidow[2]

% Referencing
% Provides \Vref and \vref to indicate where a reference is.
\usepackage{varioref} 
% Hyperlinks references
\usepackage[bookmarks=true,bookmarksopen=true]{hyperref} 
% Provides \Cref, \cref, \Vref, \vref to include the type of reference: fig/eqn/tbl
\usepackage{cleveref} 
% Setup Hyperref
\hypersetup{
  colorlinks   = true,              %Colours links instead of ugly boxes
  urlcolor     = blue,              %Colour for external hyperlinks
  linkcolor    = blue,              %Colour of internal links
  citecolor    = blue                %Colour of citations
}
% Names for Clever Ref
\crefname{table}{table}{tables}
\Crefname{table}{Table}{Tables}
\crefname{figure}{figure}{figures}
\Crefname{figure}{Figure}{Figures}
\crefname{equation}{equation}{equations}
\Crefname{equation}{Equation}{Equations}

% Wits Citation Style
\usepackage{natbib} \input{natbib-add}
\bibliographystyle{named-wits}
\bibpunct{[}{]}{;}{a}{}{}  % to get correct punctuation for bibliography
\setlength{\skip\footins}{1.5cm}
\newcommand{\citets}[1]{\citeauthor{#1}'s \citeyearpar{#1}}
\renewcommand\bibname{References}  

\pagestyle{headings}

\pagestyle{plain}
\pagenumbering{roman}

\renewenvironment{abstract}{\ \vfill\begin{center}\textbf{Abstract}\end{center}\addcontentsline{toc}{section}{Abstract}}{\vfill\vfill\newpage}
\newenvironment{declaration}{\ \vfill\begin{center}\textbf{Declaration}\end{center}\addcontentsline{toc}{section}{Declaration}}{\vfill\vfill\newpage}
\newenvironment{acknowledgements}{\ \vfill\begin{center}\textbf{Acknowledgements}\end{center}\addcontentsline{toc}{section}{Acknowledgements}}{\vfill\vfill\newpage}

\begin{document}
\onecolumn
\thispagestyle{empty}

\setcounter{page}{0}
\addcontentsline{toc}{chapter}{Preface}
\ 
\begin{center}
  \vfill
  {
  \huge \bf \textsc{Automated Grading of Free-Text Student Submissions Using Large Language Models.}\\
  \large Subtitle\\[20pt]
  \large School of Computer Science \& Applied Mathematics\\
  \large University of the Witwatersrand\\[20pt]
  \normalsize
  Sphamandla Mbuyazi\\
  2618115\\[20pt]
  Supervised by Prof. Richard Klein\\[10pt]
  \today
  }


  \vfill
  \vfill
  \includegraphics[width=1.5cm]{images/wits}
  \vspace{10pt}\\
  \small{Ethics Clearance Number: XX/XX/XX}\\[10pt]
  \small{A proposal submitted to the Faculty of Science, University of the Witwatersrand, Johannesburg,
in partial fulfilment of the requirements for the degree of Bachelor of Science with Honours}\\
\end{center}
\vfill
\newpage

\pagestyle{plain}
\setcounter{page}{1}

\phantomsection
\begin{abstract}
Abstract things....
\end{abstract}

\phantomsection
\begin{declaration}
I, ---------, hereby declare the contents of this research proposal to be my own work.
This proposal is submitted for the degree of Bachelor of Science with Honours in Computer Science at the University of the Witwatersrand.
This work has not been submitted to any other university, or for any other degree.
\end{declaration}

\phantomsection
\begin{acknowledgements}
Thanks World.
\end{acknowledgements}


\phantomsection
\addcontentsline{toc}{section}{Table of Contents}
\tableofcontents
\newpage
\phantomsection
\addcontentsline{toc}{section}{List of Figures}
\listoffigures
\newpage
\phantomsection
\addcontentsline{toc}{section}{List of Tables}
\listoftables
\newpage
\pagenumbering{arabic}

\chapter{Introduction}
\section{Hello World}
Introduction things...\\
\section{Another Section}
\subsection{This is a Subsection}
\subsubsection{This is a subsubsection}
This is just a paragraph
\subsection{A Subsection about Citation Style}
Citations are important. Citation style for Computer Science is:
\begin{itemize}
\item When used in the text, use the authors with the date in brackets:\\ \citet{klein17} say very important things.
\item When used as a reference after a face, put everything in brackets:\\ Import things are true \citep{klein17}.
\end{itemize}

\subsection{Compiling}
Remember to compile multiple times to resolve references. Usually:
\begin{verbatim}
pdflatex file.tex
bibtex file
pdflatex file.tex
pdflatex file.tex
\end{verbatim}


\chapter{Floats}
\LaTeX\ decides how to place images. It also does the referencing for you as seen in \Cref{fig:thing1}. If you have subimages, they should have their own captions and labels -- look into the subfig or subfigure packages.

\begin{figure}[ht]
	\centering
	\includegraphics[width=0.1\linewidth]{images/wits}
	\caption{This is an image}
	\label{fig:thing1}
\end{figure}

Figure captions are at the bottom. Table title are at the top of the table as seen in \Vref{tab:tab1}. There is a package called BookTabs which is \textit{way} better for tables and you should learn how to use that instead.

\begin{table}[p]
	\centering
	\caption{Table Name}
	\label{tab:tab1}
\begin{tabular}{cc}
	\hline
	Col1 & Col2\\
	\hline\hline 
	R0,C0 & R0,C1 \\ 
	R1,C0 & R1,C1 \\ 
	\hline
\end{tabular} 
\end{table}

Usually let \LaTeX\ handle the placement of floats unless you \textit{really} need to force it to do something else. The \texttt{float} package used above allows you to use \texttt{H} as the placement which means \textit{here and only here}. When using the float package, the placement options are:
\begin{enumerate}
\item h -- a gentle nudge to place it here if possible
\item t -- top of a page
\item b -- bottom of a page
\item H -- here and only here, do not move it at all
\item p -- on its own page
\end{enumerate}


\chapter{Some Referencing Tricks}
CleverRef and VarioRef are helpful:
\begin{itemize}
	\item Normal Ref: See Figure \ref{fig:thing1}
	\item CleverRef: See \Cref{fig:thing1} and \Cref{tab:tab1}
	\item CleverRef+VarioRef: See \Vref{fig:thing1} and \Vref{tab:tab1}
\end{itemize}

\chapter{IDE/Editors}
Overleaf has a great online editor for latex. Use it. 

\appendix
\chapter{Extra Stuff}\label{app:extra}
\section{What is an appendix?}\label{app:whatis}

An appendix is useful when there is information that you need to include, but breaks the flow of your document, e.g. a large number of figures/tables may need to be shown, but maybe only one needs to be in the text and the rest are just included for completeness.

\nocite{*}

\bibliography{references}\addcontentsline{toc}{chapter}{References}
\end{document}
